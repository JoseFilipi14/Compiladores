\section{Construção de árvores sintáticas}

\begin{frame}[fragile]{Construção de árvores sintáticas para expressões}

    \begin{itemize}
        \item Árvores sintáticas para expressões podem ser construídas de forma semelhante à tradução para notação posfixa
       %\pause

        \item Deve ser construído um nó para cada operação e cada operando
       %\pause

        \item Os filhos do nó de um operador serão subárvores que representam as subexpressões que constituem os operandos daquele operador
       %\pause

        \item Cada nó pode ser implementado como um registro com vários campos que caracterizam o nó
       %\pause

        \item O registro de nós que representam operadores devem conter um campo que identifica o operador e os demais campos devem ser ponteiros para os
            operandos
       %\pause

        \item As folhas das árvores contém os tokens
       %\pause

        \item O registro de uma folha deve identificar o token e também armazenar um ponteiro para a entrada do token na tabela de símbolos
    \end{itemize}

\end{frame}

\begin{frame}[fragile]{Funções para a criação de nós da árvore sintática de uma expressão}

    Cada uma das funções abaixo retorna um ponteiro para o nó criado. Assuma que os operadores são todos binários.
    \vspace{0.2in}
   %\pause

    \begin{enumerate}
        \item \Call{criarNo}{$op, L, R$}: cria um nó de operador cujo rótulo é $op$, $L$ é o ponteiro do operando à esquerda e $R$ o ponteiro do operando à
            direita
       %\pause

        \item \Call{criarFolha}{\textbf{id}, $p$}: cria um nó para um identificador com rótulo \textbf{id}, onde $p$ é o ponteiro para o identificador na tabela 
            de símbolos
       %\pause

        \item \Call{criarFolha}{\textbf{num}, $val$}: cria um nó para um número, com rótulo \textbf{num}, cujo valor é indicado por $val$
    \end{enumerate}

\end{frame}

\input{tree}

\begin{frame}[fragile]{Definição dirigida pela sintaxe para a construção de árvores sintáticas}

    \begin{itemize}
        \item É possível construir árvores sintáticas para expressões por meio de uma definição S-atribuída
       %\pause

        \item As regras semânticas agendam as chamadas das funções de criação de nós que irão construir a árvore
       %\pause

        \item O atributo sintetizado $nptr$ controla os ponteiros para os nós retornados pelas funções
       %\pause

        \item O atributo $entrada$ armazena o endereço de um token na tabela de símbolos e o atributo $val$ o valor de um número
       %\pause

        \item Estes dois atributos devem ser computados na análise léxica
    \end{itemize}

\end{frame}

\begin{frame}[fragile]{Definição dirigida pela sintaxe para expressões aritméticas de adição e subtração}

    \begin{table}
        \centering
        \begin{tabular}{lp{2cm}l}
        \toprule
        \textbf{Produção} & & \textbf{Regra semântica} \\
        \midrule
        $E\to E_1\ \code{apl}{+}\ T$ & & $E.nptr := \Call{criarNo}{\code{apl}{+}, E_1.nptr, T.nptr}$ \\
        \rowcolor[gray]{0.9}
        $E\to E_1\ \code{apl}{-}\ T$ & & $E.nptr := \Call{criarNo}{\code{apl}{-}, E_1.nptr, T.nptr}$ \\
        $E\to T$ & & $E.nptr := T.nptr$ \\
        \rowcolor[gray]{0.9}
        $T\to (E)$ & & $T.nptr := E.nptr$ \\
        $T\to \textbf{id}$ & & $T.nptr := \Call{criarNo}{\textbf{id}, \textbf{id}.entrada}$ \\
        \rowcolor[gray]{0.9}
        $T\to \textbf{num}$ & & $T.nptr := \Call{criarNo}{\textbf{num}, \textbf{num}.val}$ \\
        \bottomrule
        \end{tabular} 
    \end{table}

\end{frame}

\begin{frame}[fragile]{DAG}

    \begin{itemize}
        \item Um grafo direcionado acíclico (\textit{directed acyclic graph -- DAG}) é um grafo cujas arestas são direcionadas e que não possui ciclos
       %\pause

        \item Um DAG pode ser usado para identificar subexpressões comuns em uma expressão
       %\pause

        \item De forma similar às árvores sintáticas, um nó representa um operador e seus filhos representam os operandos
       %\pause

        \item Se houver uma ou mais expressões comuns, os nós do DAG podem ter ``mais de um pai''
       %\pause

        \item Nas árvores sintáticas, expressões comuns são duplicadas na árvore
    \end{itemize}

\end{frame}

\begin{frame}[fragile]{Construção do DAG a partir de uma definição S-atribuída}

    \begin{itemize}
        \item Uma definição S-atribuída para a construção de árvores sintáticas para expressões aritméticas de adições e subtrações pode se adaptada para a 
            construção do DAG
       %\pause

        \item De fato, basta modificar o comportamento das funções \Call{criarNo}{\ } e \Call{criarFolha}{\ }
       %\pause

        \item Ao invés de criar um novo nó a cada chamada, estas funções devem verificar se os parâmatros passados já não foram usados para construir um nó
       %\pause

        \item Em caso afirmativo, as funções devem retornar o ponteiro usado anteriormente na criação do nó
       %\pause

        \item Caso contrário, deve ser criado um novo nó e o ponteiro criado deve ser armazenado em uma tabela, associado aos parâmetros usados, para consulta
            posterior
    \end{itemize}

\end{frame}

\input{dag}
